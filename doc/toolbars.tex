<title>The Toolbars</title>

<para>
All toolbars can be moved and docked to a more convenient location (left, right or bottom sides of the application window) or on the desktop (outside the main window) by drag-and-drop, using their left side handle. The toolbars are automatically enabled/disabled depending on the currently active window: for example if the current window is a table, the Table toolbar will be enabled and all the other toolbars will be automatically disabled. 
</para>
<para>
The same approach is used for showing/hiding the toolbars: if there are no more visible tables in the workspace, the Table toolbar will be automatically hidden and will be shown again when the users adds a new table into the project. A toolbar can be manually shown/hidden by the user, at any time, by right-clicking on the main window menu area and checking/unchecking the corresponding box in the pop-up menu. 
</para>

<!--
***********************************************************************************

			Toolbar File

***********************************************************************************
-->
<sect1 id="sec-file-toolbar">
<title>The File Toolbar</title>

<para>
The <emphasis>File Toolbar</emphasis> allows to access commands mainly from the &file-menu-lnk;. Refer to this section for a more complete description of these commands.
</para>

<figure id="fig-file-toolbar">
  <title>The &appname; File Toolbar</title>
  <mediaobject> 
    <imageobject>
      <imagedata  format="PNG" fileref="pics/file-toolbar.png"/>
    </imageobject>
  </mediaobject>
</figure>

<table tocentry="1" pgwide="1" frame="sides">
<title>File toolbar commands.</title>
<tgroup cols="4">
  <colspec colname="icon" colwidth="1*" align="center"/>
  <colspec colname="command" colwidth="5*" align="left"/>
  <colspec colname="key" colwidth="2*" align="center"/>
  <colspec colname="description" colwidth="10*" align="justify"/>
  <thead>
    <row>
      <entry>Icon</entry>
      <entry>Command</entry>
      <entry>Key</entry>
      <entry>Description</entry>
    </row>
  </thead>
 <tbody>
  <row>
   <entry>&new-project-icon;</entry>
   <entry>&new-project-lnk;</entry>
   <entry>&new-project-key;</entry>
   <entry>Create a new project.</entry>
  </row>
  <row>
   <entry>&new-icon;</entry>
   <entry>&new-lnk;</entry>
   <entry></entry>
   <entry>Access to the <emphasis>New</emphasis> sub-menu.</entry>
  </row>
  <row>
   <entry>&new-table-icon;</entry>
   <entry>&new-table-lnk;</entry>
   <entry>&new-table-key;</entry>
   <entry>Create a new table.</entry>
  </row>
  <row>
   <entry>&new-matrix-icon;</entry>
   <entry>&new-matrix-lnk;</entry>
   <entry>&new-matrix-key;</entry>
   <entry>Create a new matrix.</entry>
  </row>
  <row>
   <entry>&new-note-icon;</entry>
   <entry>&new-note-lnk;</entry>
   <entry>&new-note-key;</entry>
   <entry>Create a new note window, this object can be used as a calculator or to use scripts.</entry>
  </row>
  <row>
   <entry>&new-graph-icon;</entry>
   <entry>&new-graph-lnk;</entry>
   <entry>&new-graph-key;</entry>
   <entry>Create a new empty 2D plot.</entry>
  </row>
  <row>
   <entry>&new-function-plot-icon;</entry>
   <entry>&new-function-plot-lnk;</entry>
   <entry>&new-function-plot-key;</entry>
   <entry>Creates a new plot based on a function Y=f(X).</entry>
  </row>
  <row>
   <entry>&new-surface-3d-plot-icon;</entry>
   <entry>&new-surface-3d-plot-lnk;</entry>
   <entry>&new-surface-3d-plot-key;</entry>
   <entry>Creates a new 3D plot based on a function Z=f(X,Y).</entry>
  </row>
  <row>
   <entry>&open-icon;</entry>
   <entry>&open-lnk;</entry>
   <entry>&open-key;</entry>
   <entry>Opens an existing &appname; project file.</entry>
  </row>
  <row>
   <entry>&open-template-icon;</entry>
   <entry>&open-template-lnk;</entry>
   <entry></entry>
   <entry>Opens an existing template &appname; project file.</entry>
  </row>
   <row>
   <entry>&save-project-icon;</entry>
   <entry>&save-project-lnk;</entry>
   <entry>&save-project-key;</entry>
   <entry>Saves the current project.</entry>
  </row>
   <row>
   <entry>&save-as-template-icon;</entry>
   <entry>&save-as-template-lnk;</entry>
   <entry></entry>
   <entry>Saves the current project as a template.</entry>
  </row>
  <row>
   <entry>&import-ascii-icon;</entry>
   <entry>&import-ascii-lnk;</entry>
   <entry></entry>
   <entry>Imports an ASCII file into one or multiple tables.</entry>
  </row>
<!--			removed in 0.20
  <row>
   <entry>&duplicate-window-icon;</entry>
   <entry>&duplicate-window-lnk;</entry>
   <entry></entry>
   <entry>Clonates the active window.</entry>
  </row>
-->
  <row>
   <entry>&print-icon;</entry>
   <entry>&print-lnk;</entry>
   <entry>&print-key;</entry>
   <entry>Print the active window.</entry>
  </row>
  <row>
   <entry>&export-to-pdf-icon;</entry>
   <entry>&export-to-pdf-lnk;</entry>
   <entry></entry>
   <entry>Export to PDF.</entry>
  </row>
  <row>
   <entry>&project-explorer-icon;</entry>
   <entry>&project-explorer-lnk;</entry>
   <entry>&project-explorer-key;</entry>
   <entry>Show or hide the project explorer.</entry>
  </row>
  <row>
   <entry>&results-log-icon;</entry>
   <entry>&results-log-lnk;</entry>
   <entry></entry>
   <entry>Show or hide the results window.</entry>
  </row>
</tbody>
</tgroup>
</table>
</sect1>
<!--
***********************************************************************************

			Toolbar Edit

***********************************************************************************
-->
<sect1 id="sec-edit-toolbar">
<title>The Edit Toolbar</title>

<figure id="fig-edit-toolbar">
  <title>The &appname; Edit Toolbar</title>
  <mediaobject> 
    <imageobject>
      <imagedata  format="PNG" fileref="pics/edit-toolbar.png"/>
    </imageobject>
  </mediaobject>
</figure>

<table tocentry="1" pgwide="1" frame="border">
<title>Edit toolbar commands.</title>
<tgroup cols="4">
  <colspec colname="icon" colwidth="1*" align="center"/>
  <colspec colname="command" colwidth="5*" align="left"/>
  <colspec colname="key" colwidth="2*" align="center"/>
  <colspec colname="description" colwidth="10*" align="justify"/>
  <thead>
    <row>
      <entry>Icon</entry>
      <entry>Command</entry>
      <entry>Key</entry>
      <entry>Description</entry>
    </row>
  </thead>
 <tbody>
  <row>
   <entry>&undo-icon;</entry>
   <entry>&undo-lnk;</entry>
   <entry>&undo-key;</entry>
   <entry>Undo the last command, this feature doesn't work for plot modifications.</entry>
  </row>
  <row>
   <entry>&redo-icon;</entry>
   <entry>&redo-lnk;</entry>
   <entry>&redo-key;</entry>
   <entry>Redo the last command, this feature doesn't work for plot modifications.</entry>
  </row>
  <row>
   <entry>&cut-icon;</entry>
   <entry>&cut-lnk;</entry>
   <entry>&cut-key;</entry>
   <entry>Cut the current selection.</entry>
  </row>
  <row>
   <entry>&copy-icon;</entry>
   <entry>&copy-lnk;</entry>
   <entry>&copy-key;</entry>
   <entry>Copy the current selection.</entry>
  </row>
  <row>
   <entry>&paste-icon;</entry>
   <entry>&paste-lnk;</entry>
   <entry>&paste-key;</entry>
   <entry>Paste the current selection.</entry>
  </row>
  <row>
   <entry>&delete-icon;</entry>
   <entry>&delete-lnk;</entry>
   <entry>&delete-key;</entry>
   <entry>Delete the current selection.</entry>
  </row>
</tbody>
</tgroup>
</table>
</sect1>
<!--
***********************************************************************************

			Toolbar Plot

***********************************************************************************
-->
<sect1 id="sec-plot-toolbar">
<title>The Plot Toolbar.</title>

<para>This toolbar is only active when a table is selected. It allows the quick access to the commands of the &plot-menu-lnk; which are used for the creation of new plots.</para>

<figure id="plot-file-toolbar">
  <title>The &appname; Plot Toolbar with its different sub-menus</title>
  <mediaobject>
    <imageobject>
      <imagedata  format="PNG" fileref="pics/plot-toolbar.png"/>
    </imageobject>
  </mediaobject>
</figure>

<table tocentry="1" pgwide="1" frame="sides">
<title>Plot toolbar commands</title>
<tgroup cols="4">
  <colspec colname="icon" colwidth="1*" align="center"/>
  <colspec colname="command" colwidth="6*" align="left"/>
  <colspec colname="key" colwidth="2*" align="center"/>
  <colspec colname="description" colwidth="9*" align="justify"/>
  <thead>
    <row>
      <entry>Icon</entry>
      <entry>Command</entry>
      <entry>Key</entry>
      <entry>Description</entry>
    </row>
  </thead>
 <tbody>
  <row>
   <entry>&line-symbol-icon;</entry>
   <entry namest="command" nameend="description" align="center"><emphasis>Access to the submenus for lines/Symbol plot types.</emphasis></entry>
  </row>
  <row>
   <entry>&line-icon;</entry>
   <entry>&line-lnk;</entry>
   <entry></entry>
   <entry>Build a graph with data plotted as lines</entry>
  </row>
  <row>
   <entry>&scatter-icon;</entry>
   <entry>&scatter-lnk;</entry>
   <entry></entry>
   <entry>Build a graph with data plotted as scatter of points</entry>
  </row>
  <row>
   <entry>&line-symbol-icon;</entry>
   <entry>&line-symbol-lnk;</entry>
   <entry></entry>
   <entry>Build a graph with data plotted as lines with symbols</entry>
  </row>
  <row>
   <entry>&spline-icon;</entry>
   <entry>&spline-lnk;</entry>
   <entry></entry>
   <entry>Build a graph with data plotted as smoothed lines</entry>
  </row>
  <row>
   <entry>&vertical-drop-lines-icon;</entry>
   <entry>&vertical-drop-lines-lnk;</entry>
   <entry></entry>
   <entry>Build a graph with data plotted as vertical drop lines</entry>
  </row>
  <row>
   <entry>&horizontal-steps-icon;</entry>
   <entry>&horizontal-steps-lnk;</entry>
   <entry></entry>
   <entry>Build a graph with data plotted as horizontal step lines</entry>
  </row>
  <row>
   <entry>&vertical-steps-icon;</entry>
   <entry>&vertical-steps-lnk;</entry>
   <entry></entry>
   <entry>Build a graph with data plotted as vertical step lines</entry>
  </row>
  <row>
   <entry>&columns-icon;</entry>
   <entry namest="command" nameend="description" align="center"><emphasis>Access to the sub-menu for columns and rows plots</emphasis></entry>
  </row>
   <row>
   <entry>&columns-icon;</entry>
   <entry>&columns-lnk;</entry>
   <entry></entry>
   <entry>Build a graph with data plotted as columns</entry>
  </row>
 <row>
   <entry>&rows-icon;</entry>
   <entry>&rows-lnk;</entry>
   <entry></entry>
   <entry>Build a graph with data plotted as rows</entry>
  </row>
  <row>
   <entry>&area-icon;</entry>
   <entry>&area-lnk;</entry>
   <entry></entry>
   <entry>Build a graph with data plotted as lines with a filling of areas.</entry>
  </row>
  <row>
   <entry>&histogram-icon;</entry>
   <entry>&histogram-lnk;</entry>
   <entry></entry>
   <entry>Build a graph with data plotted as an histogram.</entry>
  </row>
  <row>
   <entry>&box-icon;</entry>
   <entry>&box-lnk;</entry>
   <entry></entry>
   <entry>Build a graph with data plotted as an histogram.</entry>
  </row>
  <row>
   <entry>&vectors-xyxy-icon;</entry>
   <entry namest="command" nameend="description" align="center"><emphasis>access to the sub-menu for vector plots.</emphasis></entry>
  </row>
  <row>
   <entry>&vectors-xyxy-icon;</entry>
   <entry>&vectors-xyxy-lnk;</entry>
   <entry></entry>
   <entry>Build a graph with data plotted as vectors defined by two points.</entry>
  </row>
  <row>
   <entry>&vectors-xyam-icon;</entry>
   <entry>&vectors-xyam-lnk;</entry>
   <entry></entry>
   <entry>Build a graph with data plotted as vectors defined by an origin and a direction.</entry>
  </row>
</tbody>
</tgroup>
</table>
</sect1>
<!--
***********************************************************************************

			Toolbar Graph

***********************************************************************************
-->
<sect1 id="sec-graph-toolbar">
<title>The Graph Toolbar.</title>

<para>This toolbar is only active when a plot window is selected. It allows the quick access to the commands of the &graph-menu-lnk; which are used for the modification of the plots and of the data points of the plots.</para>
<figure id="graph-file-toolbar">
  <title>The &appname; Graph Toolbar with its different sub-menus</title>
  <mediaobject>
    <imageobject>
      <imagedata  format="PNG" fileref="pics/graph-toolbar.png"/>
    </imageobject>
  </mediaobject>
</figure>

<table tocentry="1" pgwide="1" frame="sides">
<title>Plot toolbar commands</title>
<tgroup cols="4">
  <colspec colname="icon" colwidth="1*" align="center"/>
  <colspec colname="command" colwidth="5*" align="left"/>
  <colspec colname="key" colwidth="2*" align="center"/>
  <colspec colname="description" colwidth="10*" align="justify"/>
  <thead>
    <row>
      <entry>Icon</entry>
      <entry>Command</entry>
      <entry>Key</entry>
      <entry>Description</entry>
    </row>
  </thead>
 <tbody>
  <row>
   <entry>&pointer-icon;</entry>
   <entry>&pointer-lnk;</entry>
   <entry></entry>
   <entry>Comes back to the normal pointer mode, this is useful when you have select other modes of the plot window such as the <link linkend="data-reader-cmd">data reader</link>.</entry>
  </row>

  <row>
   <entry>&arrange-layers-icon;</entry>
   <entry namest="command" nameend="description" align="center"><emphasis>Access to the layers commands.</emphasis></entry>
  </row>
  <row>
   <entry>&arrange-layers-icon;</entry>
   <entry>&arrange-layers-lnk;</entry>
   <entry>&arrange-layers-key;</entry>
   <entry>Arranges the different layers of the active plot window.</entry>
  </row>
  <row>
   <entry>&add-layer-icon;</entry>
   <entry>&add-layer-lnk;</entry>
   <entry>&add-layer-key;</entry>
   <entry>Adds a new layer to the active plot window, or remove a layer from the selected plot window.</entry>
  </row>

  <row>
   <entry>&add-remove-curve-icon;</entry>
   <entry></entry>
   <entry></entry>
   <entry><emphasis>Access to the curves sub-menu.</emphasis></entry>
  </row>
  <row>
   <entry>&add-remove-curve-icon;</entry>
   <entry>&add-remove-curve-lnk;</entry>
   <entry>&add-remove-curve-key;</entry>
   <entry>Adds or removes curves to the active plot window.</entry>
  </row>

  <row>
   <entry>&add-text-icon;</entry>
   <entry namest="command" nameend="description" align="center"><emphasis>Access to the commands for addition of graphics objects to the current plot.</emphasis></entry>
  </row>
  <row>
   <entry>&add-text-icon;</entry>
   <entry>&add-text-lnk;</entry>
   <entry>&add-text-key;</entry>
   <entry>Add a new text element in the active plot.</entry>
  </row>

  <row>
   <entry>&zoom-in-icon;</entry>
   <entry>&zoom-in-lnk;</entry>
   <entry>&zoom-in-key;</entry>
   <entry>Switches the active plot layer to the zoom mode.</entry>
  </row>
  <row>
   <entry>&zoom-out-icon;</entry>
   <entry>&zoom-out-lnk;</entry>
   <entry>&zoom-out-key;</entry>
   <entry>Switches the active plot layer to the zoom mode.</entry>
  </row>
  <row>
   <entry>&rescale-icon;</entry>
   <entry>&rescale-lnk;</entry>
   <entry>&rescale-key;</entry>
   <entry>Reset the zoom in order to show all the data.</entry>
  </row>
  <row>
   <entry>&screen-reader-icon;</entry>
   <entry>&screen-reader-lnk;</entry>
   <entry></entry>
   <entry>Switches the active plot layer to the &screen-reader-cmd; mode.</entry>
  </row>
  <row>
   <entry>&data-reader-icon;</entry>
   <entry>&data-reader-lnk;</entry>
   <entry>&data-reader-key;</entry>
   <entry>Switches the data display mode.</entry>
  </row>
  <row>
   <entry>&select-data-range-icon;</entry>
   <entry>&select-data-range-lnk;</entry>
   <entry>&select-data-range-key;</entry>
   <entry>Switches the active plot to the &select-data-range-cmd; mode.</entry>
  </row>
<!-- remove from toolbar in 0.20 -> transfer to sub menus
  <row>
   <entry>&add-error-bars-icon;</entry>
   <entry><link linkend="add-error-bars-cmd">&add-error-bars-cmd;</link></entry>
   <entry>&add-error-bars-key;</entry>
   <entry>Adds error bars to a curve of the active plot window.</entry>
  </row>
  <row>
   <entry>&add-function-icon;</entry>
   <entry><link linkend="add-function-cmd">&add-function-cmd;</link></entry>
   <entry>&add-function-key;</entry>
   <entry>Adds a curve based on a function to the active plot window.</entry>
  </row>
  <row>
   <entry>&new-legend-icon;</entry>
   <entry><link linkend="new-legend-cmd">&new-legend-cmd;</link></entry>
   <entry>&new-legend-key;</entry>
   <entry>Adds a new legend to the active plot window.</entry>
  </row>
  <row>
   <entry>&move-data-points-icon;</entry>
   <entry><link linkend="move-data-points-cmd">&move-data-points-cmd;</link></entry>
   <entry>&move-data-points-key;</entry>
   <entry>Allows to move data points on the active plot.</entry>
  </row>
  <row>
   <entry>&remove-data-points-icon;</entry>
   <entry><link linkend="remove-data-points-cmd">&remove-data-points-cmd;</link></entry>
   <entry>&remove-data-points-key;</entry>
   <entry>Allows to remove data points on the active plot.</entry>
  </row>
  <row>
   <entry>&draw-line-icon;</entry>
   <entry>&draw-line-lnk;</entry>
   <entry>&draw-line-key;</entry>
   <entry>Add a new line on the active plot.</entry>
  </row>
  <row>
   <entry>&draw-arrow-icon;</entry>
   <entry>&draw-arrow-lnk;</entry>
   <entry>&draw-arrow-key;</entry>
   <entry>Add a new arrow on the active plot.</entry>
  </row>
  <row>
   <entry>&add-time-stamp-icon;</entry>
   <entry>&add-time-stamp-lnk;</entry>
   <entry>&add-time-stamp-key;</entry>
   <entry>Add a time/date label on the active plot.</entry>
  </row>
  <row>
   <entry>&add-image-icon;</entry>
   <entry>&add-image-lnk;</entry>
   <entry>&add-image-key;</entry>
   <entry>Insert a new image in the active plot.</entry>
  </row>
-->
</tbody>
</tgroup>
</table>
</sect1>
<!--
***********************************************************************************

			Toolbar Table

***********************************************************************************
-->

<sect1 id="sec-table-toolbar">
<title>The Table Toolbar.</title>

<para>This toolbar allows a quick access to the commands of the &table-menu-lnk; used to modify a table.</para>

<figure id="table-toolbar">
  <title>The &appname; Table Toolbar</title>
  <mediaobject> 
    <imageobject>
      <imagedata  format="PNG" fileref="pics/table-toolbar.png"/>
    </imageobject>
  </mediaobject>
</figure>

<table tocentry="1" pgwide="1" frame="sides">
<title>Table toolbar commands.</title>
<tgroup cols="4">
  <colspec colname="icon" colwidth="1*" align="center"/>
  <colspec colname="command" colwidth="5*" align="left"/>
  <colspec colname="key" colwidth="2*" align="center"/>
  <colspec colname="description" colwidth="10*" align="justify"/>
  <thead>
    <row>
      <entry>Icon</entry>
      <entry>Command</entry>
      <entry>Key</entry>
      <entry>Description</entry>
    </row>
  </thead>
 <tbody>
<!--
  <row>
   <entry>&line-icon;</entry>
   <entry><link linkend="line-cmd">plot -> &line-cmd;</link></entry>
   <entry></entry>
   <entry>plot with the line style.</entry>
  </row>
  <row>
   <entry>&scatter-icon;</entry>
   <entry><link linkend="scatter-cmd">plot -> &scatter-cmd;</link></entry>
   <entry></entry>
   <entry>plot with the scatter style.</entry>
  </row>
  <row>
   <entry>&line-symbol-icon;</entry>
   <entry><link linkend="line-symbol-cmd">plot -> &line-symbol-cmd;</link></entry>
   <entry></entry>
   <entry>plot with the line+symbol style.</entry>
  </row>
  <row>
   <entry>&columns-icon;</entry>
   <entry><link linkend="columns-cmd">plot -> &columns-cmd;</link></entry>
   <entry></entry>
   <entry>plot with the columns style.</entry>
  </row>
  <row>
   <entry>&rows-icon;</entry>
   <entry><link linkend="rows-cmd">plot -> &rows-cmd;</link></entry>
   <entry></entry>
   <entry>plot with the rows style.</entry>
  </row>
  <row>
   <entry>&area-icon;</entry>
   <entry><link linkend="area-cmd">plot -> &area-cmd;</link></entry>
   <entry></entry>
   <entry>plot with the area style.</entry>
  </row>
  <row>
   <entry>&pie-icon;</entry>
   <entry><link linkend="pie-cmd">plot -> &pie-cmd;</link></entry>
   <entry></entry>
   <entry>plot with the pie style.</entry>
  </row>
  <row>
   <entry>&histogram-icon;</entry>
   <entry><link linkend="histogram-cmd">plot -> &histogram-cmd;</link></entry>
   <entry></entry>
   <entry>plot with the histogram style.</entry>
  </row>
  <row>
   <entry>&vectors-xyxy-icon;</entry>
   <entry><link linkend="vectors-xyxy-cmd">plot -> &vectors-xyxy-cmd;</link></entry>
   <entry></entry>
   <entry>plot with the vector style.</entry>
  </row>
  <row>
   <entry>&ribbons-icon;</entry>
   <entry><link linkend="ribbons-cmd">plot -> &ribbons-cmd;</link></entry>
   <entry></entry>
   <entry>plot with the 3D ribbons style.</entry>
  </row>
  <row>
   <entry>&bars-icon;</entry>
   <entry><link linkend="bars-cmd">plot -> &bars-cmd;</link></entry>
   <entry></entry>
   <entry>plot with the 3D bars style.</entry>
  </row>
  <row>
   <entry>&scatter3d-icon;</entry>
   <entry><link linkend="scatter3d-cmd">plot -> &scatter3d-cmd;</link></entry>
   <entry></entry>
   <entry>plot with the 3D scatter style.</entry>
  </row>
  <row>
   <entry>&trajectory-icon;</entry>
   <entry><link linkend="trajectory-cmd">plot -> &trajectory-cmd;</link></entry>
   <entry></entry>
   <entry>plot with the trajectory style.</entry>
  </row>
-->
  <row>
   <entry>&table-dimensions-icon;</entry>
   <entry>&table-dimensions-lnk;</entry>
   <entry></entry>
   <entry>modify the number of rows and columns of the table</entry>
  </row>
  <row>
   <entry>&add-column-icon;</entry>
   <entry>&add-column-lnk;</entry>
   <entry></entry>
   <entry>add a new column to the table</entry>
  </row>
  <row>
   <entry>&statistics-on-columns-icon;</entry>
   <entry>&statistics-on-columns-lnk;</entry>
   <entry></entry>
   <entry>compute statistical parameters on selected columns</entry>
  </row>
  <row>
   <entry>&statistics-on-rows-icon;</entry>
   <entry>&statistics-on-rows-lnk;</entry>
   <entry></entry>
   <entry>compute statistical parameters on selected row</entry>
  </row>
</tbody>
</tgroup>
</table>
</sect1>
<!--
***********************************************************************************

			Toolbar Matrix Plot

***********************************************************************************
-->
<sect1 id="sec-matrix-plot-toolbar">
<title>The matrix plot Toolbar.</title>

<para></para>
<figure id="matrix-plot-toolbar">
  <title>The &appname; matrix Plot Toolbar</title>
  <mediaobject> 
    <imageobject>
      <imagedata  format="PNG" fileref="pics/plot3d-toolbar.png"/>
    </imageobject>
  </mediaobject>
</figure>

<table tocentry="1" pgwide="1" frame="sides">
<title>3D Plot toolbar commands.</title>
<tgroup cols="4">
  <colspec colname="icon" colwidth="1*" align="center"/>
  <colspec colname="command" colwidth="5*" align="left"/>
  <colspec colname="key" colwidth="2*" align="center"/>
  <colspec colname="description" colwidth="10*" align="justify"/>
  <thead>
    <row>
      <entry>Icon</entry>
      <entry>Command</entry>
      <entry>Key</entry>
      <entry>Description</entry>
    </row>
  </thead>
 <tbody>
  <row>
   <entry>&mesh-icon;</entry>
   <entry>&mesh-lnk;</entry>
   <entry></entry>
   <entry>Draw a surface with the wireframe style.</entry>
  </row>
  <row>
   <entry>&mesh-hidden-icon;</entry>
   <entry>&mesh-hidden-lnk;</entry>
   <entry></entry>
   <entry>Draw a surface with the mesh style (with hidden lines).</entry>
  </row>
  <row>
   <entry>&polygons-icon;</entry>
   <entry>&polygons-lnk;</entry>
   <entry></entry>
   <entry>Draw a surface with the polygons style.</entry>
  </row>
  <row>
   <entry>&mesh-polygons-icon;</entry>
   <entry>&mesh-polygons-lnk;</entry>
   <entry></entry>
   <entry>Draw a surface with the mesh+polygons style.</entry>
  </row>
  <row>
   <entry>&bar-style-icon;</entry>
   <entry>&bar-style-lnk;</entry>
   <entry></entry>
   <entry>Changes the styles of the bars.</entry>
  </row>
  <row>
   <entry>&scatter3d-icon;</entry>
   <entry>&scatter3d-lnk;</entry>
   <entry></entry>
   <entry>Draw data points as a clouds of points in a 3D space.</entry>
  </row>
  <row>
   <entry>&contour-color-icon;</entry>
   <entry>&contour-color-lnk;</entry>
   <entry></entry>
   <entry>Draw data points as a map with a color filling between isolines.</entry>
  </row>
  <row>
   <entry>&contour-lines-icon;</entry>
   <entry>&contour-lines-lnk;</entry>
   <entry></entry>
   <entry>Draw data points as a map with isolines.</entry>
  </row>
  <row>
   <entry>&gray-scale-icon;</entry>
   <entry>&gray-scale-lnk;</entry>
   <entry></entry>
   <entry>Draw data points as a map with a gray palette filling between isolines.</entry>
  </row>
</tbody>
</tgroup>
</table>

</sect1>

<!--
***********************************************************************************

			Toolbar 3D surfaces

***********************************************************************************
-->

<sect1 id="sec-3d-surface-toolbar">
<title>The 3D Surfaces Toolbar.</title>

<para></para>
<figure id="surfaces-3d-toolbar">
  <title>The &appname; 3D Surfaces Toolbar</title>
  <mediaobject> 
    <imageobject>
      <imagedata  format="PNG" fileref="pics/3d-surfaces-toolbar.png"/>
    </imageobject>
  </mediaobject>
</figure>

<table tocentry="1" pgwide="1" frame="sides">
<title>3D Plot toolbar commands.</title>
<tgroup cols="4">
  <colspec colname="icon" colwidth="1*" align="center"/>
  <colspec colname="command" colwidth="5*" align="left"/>
  <colspec colname="key" colwidth="2*" align="center"/>
  <colspec colname="description" colwidth="10*" align="justify"/>
  <thead>
    <row>
      <entry>Icon</entry>
      <entry>Command</entry>
      <entry>Key</entry>
      <entry>Description</entry>
    </row>
  </thead>
 <tbody>
  <row>
   <entry id="frame-cmd">&frame-icon;</entry>
   <entry>&frame-cmd;</entry>
   <entry></entry>
   <entry>Draw only the three axes.</entry>
  </row>
  <row>
   <entry id="box-cmd">&box-icon;</entry>
   <entry>&box-cmd;</entry>
   <entry></entry>
   <entry>Draw the three axes and the 3D box around the plot.</entry>
  </row>
  <row>
   <entry id="no-axes-cmd">&no-axes-icon;</entry>
   <entry>&no-axes-cmd;</entry>
   <entry></entry>
   <entry>Doesn't draw the axes nor the box.</entry>
  </row>
  <row>
   <entry id="front-grid-cmd">&front-grid-icon;</entry>
   <entry>&front-grid-cmd;</entry>
   <entry></entry>
   <entry>Draw a grid on the front panel. The position of this grid is the plan defined by y=y<subscript>min</subscript>.</entry>
  </row>
  <row>
   <entry id="back-grid-cmd">&back-grid-icon;</entry>
   <entry>&back-grid-cmd;</entry>
   <entry></entry>
   <entry>Draw a grid on the back panel. The position of this grid is the plan defined by y=y<subscript>max</subscript>.</entry>
  </row>
  <row>
   <entry id="left-grid-cmd">&left-grid-icon;</entry>
   <entry>&left-grid-cmd;</entry>
   <entry></entry>
   <entry>Draw a grid on the left panel. The position of this grid is the plan defined by x=x<subscript>min</subscript>.</entry>
  </row>
  <row>
   <entry id="right-grid-cmd">&right-grid-icon;</entry>
   <entry>&right-grid-cmd;</entry>
   <entry></entry>
   <entry>Draw a grid on the right panel. The position of this grid is the plan defined by x=x<subscript>max</subscript>.</entry>
  </row>
  <row>
   <entry id="top-grid-cmd">&top-grid-icon;</entry>
   <entry>&top-grid-cmd;</entry>
   <entry></entry>
   <entry>Draw a grid on the top panel. The position of this grid is the plan defined by z=z<subscript>max</subscript>.</entry>
  </row>
  <row>
   <entry>&floor-grid-icon;</entry>
   <entry><link linkend="floor-grid-cmd">plot -> &floor-grid-cmd;</link></entry>
   <entry></entry>
   <entry>Draw a grid on the bottom panel.</entry>
  </row>
  <row>
   <entry id="floor-grid-cmd">&floor-grid-icon;</entry>
   <entry>&floor-grid-cmd;</entry>
   <entry></entry>
   <entry>Draw a grid on the bottom panel. The position of this grid is the plan defined by z=z<subscript>min</subscript>.</entry>
  </row>
  <row>
   <entry id="perspective-cmd">&perspective-icon;</entry>
   <entry>&perspective-cmd;</entry>
   <entry></entry>
   <entry>Enables/Disables the 3D perspective mode.</entry>
  </row>
  <row>
   <entry id="reset-rotation-cmd">&reset-rotation-icon;</entry>
   <entry>&reset-rotation-cmd;</entry>
   <entry></entry>
   <entry>Resets the rotation of the 3D plot to the default values.</entry>
  </row>
  <row>
   <entry id="autoscale-cmd">&autoscale-icon;</entry>
   <entry>&autoscale-cmd;</entry>
   <entry></entry>
   <entry>Finds the best layout of the 3D plot fitting the window size. It readjusts the length of the axis ticks to a default value.</entry>
  </row>
  <row>
   <entry id="bar-style-cmd">&bar-style-icon;</entry>
   <entry>&bar-style-cmd;</entry>
   <entry></entry>
   <entry>If the active 3D plot is a <link linkend="fig-3d-bars">3D histogram</link>, this command is used to modify the style of the bars.</entry>
  </row>
  <row>
   <entry id="dots-cmd">&dots-icon;</entry>
   <entry>&dots-cmd;</entry>
   <entry></entry>
   <entry>If the active 3D plot is a <link linkend="fig-3d-scatter">3D scatter</link>, this command is used to modify the style of the data points to dots.</entry>
  </row>
  <row>
   <entry id="cones-cmd">&cones-icon;</entry>
   <entry>&cones-cmd;</entry>
   <entry></entry>
   <entry>If the active 3D plot is a <link linkend="fig-3d-scatter">3D scatter</link>, this command is used to modify the style of the data points to cones. It is then possible to modify the drawing parameters of the cones by double clicking on the plotting area.</entry>
  </row>
  <row>
   <entry id="cross-hairs-cmd">&cross-hairs-icon;</entry>
   <entry>&cross-hairs-cmd;</entry>
   <entry></entry>
   <entry>If the active 3D plot is a <link linkend="fig-3d-scatter">3D scatter</link>, this command is used to modify the style of the data points to cross-hairs. It it then possible to modify the drawing parameters of the crosses by double clicking on the plotting area.</entry>
  </row>
  <row>
   <entry id="mesh-cmd">&mesh-icon;</entry>
   <entry>&mesh-cmd;</entry>
   <entry></entry>
   <entry>If the active 3D plot is a 3D surface, this command is used to modify the style of the surface to a simple wireframe.</entry>
  </row>
  <row>
   <entry id="mesh-hidden-cmd">&mesh-hidden-icon;</entry>
   <entry>&mesh-hidden-cmd;</entry>
   <entry></entry>
   <entry>If the active 3D plot is a 3D surface, this command is used to modify the style of the surface to a wireframe. A computation of the hidden line is done.</entry>
  </row>
  <row>
   <entry id="polygons-cmd">&polygons-icon;</entry>
   <entry>&polygons-cmd;</entry>
   <entry></entry>
   <entry>If the active 3D plot is a 3D surface, this command is used to modify the style of the surface to polygons.</entry>
  </row>
  <row>
   <entry id="mesh-polygons-cmd">&mesh-polygons-icon;</entry>
   <entry>&mesh-polygons-cmd;</entry>
   <entry></entry>
   <entry>If the active 3D plot is a 3D surface, this command is used to modify the style of the surface to polygons with a mesh.</entry>
  </row>
  <row>
   <entry id="floor-cmd">&floor-icon;</entry>
   <entry>&floor-cmd;</entry>
   <entry></entry>
   <entry>If the active 3D plot is a 3D surface, this command is used to add a filled area projection of the surface on the floor of the plot.</entry>
  </row>
  <row>
   <entry id="isolines-cmd">&isolines-icon;</entry>
   <entry>&isolines-cmd;</entry>
   <entry></entry>
   <entry>If the active 3D plot is a 3D surface, this command is used to add an isoline.</entry>
  </row>
  <row>
   <entry id="empty-cmd">&empty-icon;</entry>
   <entry>&empty-cmd;</entry>
   <entry></entry>
   <entry>If the active 3D plot is a 3D surface, this command is used to remove any projection from the floor.</entry>
  </row>
  <row>
   <entry id="animation-cmd">&animation-icon;</entry>
   <entry>&animation-cmd;</entry>
   <entry></entry>
   <entry>Enables/Disables animation.</entry>
  </row>
</tbody>
</tgroup>
</table>

</sect1>
